\documentclass{article}
\title{Side-channel Attack on Cache and MMU}
\author{Li Dinghong \\SIST, Shanghaitech University \\27663262}
\usepackage[utf8]{inputenc}
\usepackage{graphicx}
\usepackage{listings}
\usepackage{tikz}
\usepackage[]{forest}
\usepackage[colorlinks,linkcolor=red]{hyperref}
\usepackage{amsmath, amsthm, amssymb}
\usepackage{subfloat}
\usepackage{pdfpages}
\usepackage{geometry}
\usepackage{indentfirst}
\usepackage{xcolor}
\usepackage{colortbl,booktabs}
\usepackage{pict2e}
\geometry{
	a4paper,
	total={170mm,257mm},
	left=20mm,
	top=10mm,
}

\begin{document}
\maketitle
\large

\section{Introduction and Background}{}

\section{Related Works}{}

\section{Methodology: How is it related to Operating Systems? }{
	There are several different ways to study this topic. For instance, we can propose a novel CPU architecture to raise cache security, or analyze side-channel attacks using information theory. If we want to study it from perspective of operating systems, however, it is better to narrow it to some specific contexts. 

	A good option is to run our models on browsers, as is used in previous works. Most likely, we will test both attack and defense models on three major browsers: Firefox, Safari and Chrome. 

	Side-channel attack is highly related to MMU, which is an important part in modern operating systems. 

	Attack models mainly exploit vulnerability of ASLR. 

	Existing defense models for AnC attack include naively disabling cache and reducing accuracy of timers. 
}

\section{Better Attack Models}{}

\section{Defense Models}{}

\section{Tradeoffs between Performance and Security on Cache}{
	We can compare the effectiveness of side-channel attacks on cache with different replacement policies. 
}

\end{document}